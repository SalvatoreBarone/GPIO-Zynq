\hypertarget{no_driver_8c-example}{\section{no\+Driver.\+c}
}
Questo è un programma di esempio per l'interfacciamento con una periferica my\+G\+P\+I\+O.\begin{DoxyAuthor}{Autore}
Salvatore Barone \href{mailto:salvator.barone@gmail.com}{\tt salvator.\+barone@gmail.\+com} 
\end{DoxyAuthor}
\begin{DoxyDate}{Data}
12 06 2017
\end{DoxyDate}
\begin{DoxyCopyright}{Copyright}
Copyright 2017 Salvatore Barone \href{mailto:salvator.barone@gmail.com}{\tt salvator.\+barone@gmail.\+com}
\end{DoxyCopyright}
This file is part of Zynq7000\+Driver\+Pack

Zynq7000\+Driver\+Pack is free software; you can redistribute it and/or modify it under the terms of the G\+N\+U General Public License as published by the Free Software Foundation; either version 3 of the License, or any later version.

Zynq7000\+Driver\+Pack is distributed in the hope that it will be useful, but W\+I\+T\+H\+O\+U\+T A\+N\+Y W\+A\+R\+R\+A\+N\+T\+Y; without even the implied warranty of M\+E\+R\+C\+H\+A\+N\+T\+A\+B\+I\+L\+I\+T\+Y or F\+I\+T\+N\+E\+S\+S F\+O\+R A P\+A\+R\+T\+I\+C\+U\+L\+A\+R P\+U\+R\+P\+O\+S\+E. See the G\+N\+U General Public License for more details.

You should have received a copy of the G\+N\+U General Public License along with this program; if not, write to the Free Software Foundation, Inc., 51 Franklin Street, Fifth Floor, Boston, M\+A 02110-\/1301, U\+S\+A.

In questo specifico esempio l'interfacciamento avviene da user-\/space, agendo direttamente sui registri di memoria, senza mediazione di altri driver, usando il device-\/file /dev/mem.


\begin{DoxyCodeInclude}

\textcolor{preprocessor}{#include <inttypes.h>}
\textcolor{preprocessor}{#include <stdio.h>}
\textcolor{preprocessor}{#include <stdlib.h>}
\textcolor{preprocessor}{#include <unistd.h>}
\textcolor{preprocessor}{#include <sys/mman.h>}
\textcolor{preprocessor}{#include <fcntl.h>}

\textcolor{preprocessor}{#include "\hyperlink{my_g_p_i_o_8h}{myGPIO.h}"}

\textcolor{keywordtype}{void} \hyperlink{no_driver_8c_a05909651fa170a63e98e3f8e13451b7b}{howto}(\textcolor{keywordtype}{void}) \{
    printf(\textcolor{stringliteral}{"Uso:\(\backslash\)n"});
    printf(\textcolor{stringliteral}{"noDriver -a gpio\_phisycal\_address -w|m <hex-value> -r\(\backslash\)n"});
    printf(\textcolor{stringliteral}{"\(\backslash\)t-m <hex-value>: scrive nel registro \(\backslash\)"mode\(\backslash\)"\(\backslash\)n"});
    printf(\textcolor{stringliteral}{"\(\backslash\)t-w <hex-value>: scrive nel registro \(\backslash\)"write\(\backslash\)"\(\backslash\)n"});
    printf(\textcolor{stringliteral}{"\(\backslash\)t-r: legge il valore del registro \(\backslash\)"read\(\backslash\)"\(\backslash\)n"});
    printf(\textcolor{stringliteral}{"I parametri possono anche essere usati assieme.\(\backslash\)n"});
\}

\textcolor{keywordtype}{int} \hyperlink{no_driver_8c_a218f8a9dfc36572bfe2230c5e2d2c776}{parse\_args}(   \textcolor{keywordtype}{int}         argc,
                \textcolor{keywordtype}{char}        **argv,
                uint32\_t    *gpio\_address,
                uint8\_t     *op\_mode,
                uint32\_t    *mode\_value,
                uint8\_t     *op\_write,
                uint32\_t    *write\_value,
                uint8\_t     *op\_read)
\{
    \textcolor{keywordtype}{int} par;
    \textcolor{keywordflow}{while}((par = getopt(argc, argv, \textcolor{stringliteral}{"a:w:m:r"})) != -1) \{
        \textcolor{keywordflow}{switch} (par) \{
        \textcolor{keywordflow}{case} \textcolor{charliteral}{'a'} :
            *gpio\_address = strtoul(optarg, NULL, 0);
            \textcolor{keywordflow}{break};
        \textcolor{keywordflow}{case} \textcolor{charliteral}{'w'} :
            *write\_value = strtoul(optarg, NULL, 0);
            *op\_write = 1;
            \textcolor{keywordflow}{break};
        \textcolor{keywordflow}{case} \textcolor{charliteral}{'m'} :
            *mode\_value = strtoul(optarg, NULL, 0);
            *op\_mode = 1;
            \textcolor{keywordflow}{break};
        \textcolor{keywordflow}{case} \textcolor{charliteral}{'r'} :
            *op\_read = 1;
            \textcolor{keywordflow}{break};
        \textcolor{keywordflow}{default} :
            printf(\textcolor{stringliteral}{"%c: parametro sconosciuto.\(\backslash\)n"}, par);
            \hyperlink{no_driver_8c_a05909651fa170a63e98e3f8e13451b7b}{howto}();
            \textcolor{keywordflow}{return} -1;
        \}
    \}
    \textcolor{keywordflow}{return} 0;
\}


\textcolor{keywordtype}{void} \hyperlink{no_driver_8c_a879d8b839631449ecb5bc4d0721432b6}{gpio\_op} (   \textcolor{keywordtype}{void}*       vrt\_gpio\_addr,
                uint8\_t     op\_mode,
                uint32\_t    mode\_value,
                uint8\_t     op\_write,
                uint32\_t    write\_value,
                uint8\_t     op\_read)
\{
    printf(\textcolor{stringliteral}{"Indirizzo gpio: %08x\(\backslash\)n"}, (uint32\_t)vrt\_gpio\_addr);
\textcolor{preprocessor}{#ifdef \_\_XIL\_GPIO\_\_}
\textcolor{preprocessor}{#define MODE\_OFFSET     4U}
\textcolor{preprocessor}{#define WRITE\_OFFSET    0U}
\textcolor{preprocessor}{#define READ\_OFFSET     8U}
\textcolor{preprocessor}{#else}
    \hyperlink{structmy_g_p_i_o__t}{myGPIO\_t} gpio;
    \hyperlink{group__bare-metal_ga588201358d1633c53535b288c9198531}{myGPIO\_Init}(&gpio, (uint32\_t)vrt\_gpio\_addr);
\textcolor{preprocessor}{#endif}

    \textcolor{keywordflow}{if} (op\_mode == 1) \{
\textcolor{preprocessor}{#ifdef \_\_XIL\_GPIO\_\_}
        *((uint32\_t*)(vrt\_gpio\_addr+\hyperlink{mygpiok_8c_a7a63b24ca5489eb3206598e3d90fe19c}{MODE\_OFFSET})) = mode\_value;
        mode\_value = *((uint32\_t*)(vrt\_gpio\_addr+\hyperlink{mygpiok_8c_a7a63b24ca5489eb3206598e3d90fe19c}{MODE\_OFFSET}));
\textcolor{preprocessor}{#else}
        \hyperlink{group__bare-metal_ga43e82eb0febd452635a438fbd9cb853b}{myGPIO\_SetMode}(&gpio, mode\_value, \hyperlink{group__bare-metal_gga76b849f0e0c05e7f9161bdb33396f2b1a2d66976280eb7595a42c631683bdfad6}{myGPIO\_write});
        \hyperlink{group__bare-metal_ga43e82eb0febd452635a438fbd9cb853b}{myGPIO\_SetMode}(&gpio, ~mode\_value, \hyperlink{group__bare-metal_gga76b849f0e0c05e7f9161bdb33396f2b1a1e6dc78e7641e878cadc842d39605d5d}{myGPIO\_read});
\textcolor{preprocessor}{#endif}
        printf(\textcolor{stringliteral}{"Scrittura sul registro mode: %08x\(\backslash\)n"}, mode\_value);
    \}
    \textcolor{keywordflow}{if} (op\_write == 1) \{
\textcolor{preprocessor}{#ifdef \_\_XIL\_GPIO\_\_}
        *((uint32\_t*)(vrt\_gpio\_addr+\hyperlink{mygpiok_8c_a77d96306ed0e813f93c4c3f98b970b86}{WRITE\_OFFSET})) = write\_value;
        write\_value = *((uint32\_t*)(vrt\_gpio\_addr+\hyperlink{mygpiok_8c_a77d96306ed0e813f93c4c3f98b970b86}{WRITE\_OFFSET}));
\textcolor{preprocessor}{#else}
        \hyperlink{group__bare-metal_ga9d9ce9d2db7d77a588da4a3749f2f24d}{myGPIO\_SetValue}(&gpio, write\_value, \hyperlink{group__bare-metal_ggaf634fe4a0e1eab8da5000b72d6ad362ba10d296f3711d01189cc6c2d87f7c9149}{myGPIO\_set});
        \hyperlink{group__bare-metal_ga9d9ce9d2db7d77a588da4a3749f2f24d}{myGPIO\_SetValue}(&gpio, ~write\_value, \hyperlink{group__bare-metal_ggaf634fe4a0e1eab8da5000b72d6ad362ba98cde80dbda025bd1ae7231c76b55674}{myGPIO\_reset});
\textcolor{preprocessor}{#endif}
        printf(\textcolor{stringliteral}{"Scrittura sul registro write: %08x\(\backslash\)n"}, write\_value);
    \}
    \textcolor{keywordflow}{if} (op\_read == 1) \{
        uint32\_t read\_value = 0;
\textcolor{preprocessor}{#ifdef \_\_XIL\_GPIO\_\_}
        read\_value = *((uint32\_t*)(vrt\_gpio\_addr+\hyperlink{mygpiok_8c_ad32c3a5b42163e171daccde5b5d5de02}{READ\_OFFSET}));
\textcolor{preprocessor}{#else}
        read\_value = \hyperlink{group__bare-metal_gac35776cd6652f7b932a132f3f6959a11}{myGPIO\_GetRead}(&gpio);
\textcolor{preprocessor}{#endif}
        printf(\textcolor{stringliteral}{"Lettura dat registro read: %08x\(\backslash\)n"}, read\_value);
    \}
\}



\textcolor{keywordtype}{int} \hyperlink{no_driver_8c_a3c04138a5bfe5d72780bb7e82a18e627}{main}(\textcolor{keywordtype}{int} argc, \textcolor{keywordtype}{char}** argv) \{
    uint32\_t gpio\_addr = 0;     \textcolor{comment}{// indirizzo di memoria del device gpio}
    uint8\_t op\_mode = 0;        \textcolor{comment}{// impostato ad 1 se l'utente intende effettuare scrittuara su mode}
    uint32\_t mode\_value;        \textcolor{comment}{// valore che l'utente intende scrivere nel registro mode}
    uint8\_t op\_write = 0;       \textcolor{comment}{// impostato ad 1 se l'utente intende effettuare scrittuara su write}
    uint32\_t write\_value;       \textcolor{comment}{// valore che l'utente intende scrivere nel registro write}
    uint8\_t op\_read = 0;        \textcolor{comment}{// impostato ad 1 se l'utente intende effettuare lettura da read}

    printf(\textcolor{stringliteral}{"%s build %d\(\backslash\)n"}, argv[0], BUILD); \textcolor{comment}{// BUILD viene definita in compilazione}

    \textcolor{keywordflow}{if} (\hyperlink{no_driver_8c_a218f8a9dfc36572bfe2230c5e2d2c776}{parse\_args}(argc, argv, &gpio\_addr, &op\_mode, &mode\_value, &op\_write, &write\_value, &
      op\_read) == -1)
        \textcolor{keywordflow}{return} -1;
    \textcolor{keywordflow}{if} (gpio\_addr == 0) \{
        printf(\textcolor{stringliteral}{"è necessario specificare l'indirizzo di memoria del device.\(\backslash\)n"});
        \hyperlink{no_driver_8c_a05909651fa170a63e98e3f8e13451b7b}{howto}();
        \textcolor{keywordflow}{return} -1;
    \}

    \textcolor{keywordtype}{int} descriptor = open (\textcolor{stringliteral}{"/dev/mem"}, O\_RDWR);
    \textcolor{keywordflow}{if} (descriptor < 1) \{
        perror(argv[0]);
        \textcolor{keywordflow}{return} -1;
    \}

    uint32\_t page\_size = sysconf(\_SC\_PAGESIZE);     \textcolor{comment}{// dimensione della pagina}
    uint32\_t page\_mask = ~(page\_size-1);            \textcolor{comment}{// maschera di conversione indirizzo -> indirizzo
       pagina}
    uint32\_t page\_addr = gpio\_addr & page\_mask;     \textcolor{comment}{// indirizzo della "pagina fisica" a cui è mappato il
       device}
    uint32\_t offset = gpio\_addr - page\_addr;        \textcolor{comment}{// offset del device rispetto all'indirizzo della
       pagina}
\textcolor{comment}{}    \textcolor{keywordtype}{void}* vrt\_page\_addr = mmap(NULL, page\_size, PROT\_READ | PROT\_WRITE, MAP\_SHARED, descriptor, page\_addr);
    \textcolor{keywordflow}{if} (vrt\_page\_addr == MAP\_FAILED) \{
        printf(\textcolor{stringliteral}{"Mapping indirizzo fisico - indirizzo virtuale FALLITO!\(\backslash\)n"});
        \textcolor{keywordflow}{return} -1;
    \}
    \textcolor{keywordtype}{void}* vrt\_gpio\_addr = vrt\_page\_addr + offset;   \textcolor{comment}{// indirizzo virtuale del device gpio}
\textcolor{comment}{}    \hyperlink{no_driver_8c_a879d8b839631449ecb5bc4d0721432b6}{gpio\_op}(vrt\_gpio\_addr, op\_mode, mode\_value, op\_write, write\_value, op\_read);

    munmap(vrt\_page\_addr, page\_size);
    close(descriptor);

    \textcolor{keywordflow}{return} 0;
\}


\end{DoxyCodeInclude}
 