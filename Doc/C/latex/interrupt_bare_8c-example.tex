\hypertarget{interrupt_bare_8c-example}{\section{interrupt\+\_\+bare.\+c}
}
Uso del driver my\+G\+P\+I\+O con interruzioni bare-\/metal su Zynq-\/7020 \begin{DoxyAuthor}{Autore}
Salvatore Barone \href{mailto:salvator.barone@gmail.com}{\tt salvator.\+barone@gmail.\+com} 
\end{DoxyAuthor}
\begin{DoxyDate}{Data}
23 06 2017
\end{DoxyDate}
\begin{DoxyCopyright}{Copyright}
Copyright 2017 Salvatore Barone \href{mailto:salvator.barone@gmail.com}{\tt salvator.\+barone@gmail.\+com}
\end{DoxyCopyright}
This file is part of Zynq7000\+Driver\+Pack

Zynq7000\+Driver\+Pack is free software; you can redistribute it and/or modify it under the terms of the G\+N\+U General Public License as published by the Free Software Foundation; either version 3 of the License, or any later version.

Zynq7000\+Driver\+Pack is distributed in the hope that it will be useful, but W\+I\+T\+H\+O\+U\+T A\+N\+Y W\+A\+R\+R\+A\+N\+T\+Y; without even the implied warranty of M\+E\+R\+C\+H\+A\+N\+T\+A\+B\+I\+L\+I\+T\+Y or F\+I\+T\+N\+E\+S\+S F\+O\+R A P\+A\+R\+T\+I\+C\+U\+L\+A\+R P\+U\+R\+P\+O\+S\+E. See the G\+N\+U General Public License for more details.

You should have received a copy of the G\+N\+U General Public License along with this program; if not, write to the Free Software Foundation, Inc., 51 Franklin Street, Fifth Floor, Boston, M\+A 02110-\/1301, U\+S\+A.

\paragraph*{Configurazione hardware}

L'esempio fa riferimento ad una configurazione hardware in cui, oltre alla ip-\/core Zynq7000 processing sysyem, sono presenti tre diversi device my\+G\+P\+I\+O, uno connesso ai led (base address 0x43c00000), uno connesso ai button (base address 0x43c10000) ed uno connesso agli switch (base address 0x43c20000).

\subparagraph*{I\+S\+R per la gestione di interrupt provenienti dal gpio connesso agli switch}


\begin{DoxyCode}
\textcolor{keywordtype}{void} \hyperlink{interrupt__bare_8c_ad05dc46b6c6da383d687c5116864b4ed}{swc\_isr}(\textcolor{keywordtype}{void}* data) \{
    \hyperlink{group__bare-metal_gaacca2871ac57a166e62bf431a2da7548}{myGPIO\_GlobalInterruptDisable}(&\hyperlink{interrupt__bare_8c_ac2f1233c00752afeb2ecfd305c5fe36a}{swc\_gpio});
    \hyperlink{group__bare-metal_ga402a0d20afc0cb7c25554b8b023f4253}{myGPIO\_mask} enabledInterrut = \hyperlink{group__bare-metal_ga80ef1cf3e9bd8bfd4d849a0f3b8e7b2c}{myGPIO\_EnabledPinInterrupt}(&
      \hyperlink{interrupt__bare_8c_ac2f1233c00752afeb2ecfd305c5fe36a}{swc\_gpio});
    \hyperlink{group__bare-metal_ga37d3df33ac50387d6f2e1fb5e2b13e49}{myGPIO\_PinInterruptDisable}(&\hyperlink{interrupt__bare_8c_ac2f1233c00752afeb2ecfd305c5fe36a}{swc\_gpio}, enabledInterrut);

    \hyperlink{group__bare-metal_ga402a0d20afc0cb7c25554b8b023f4253}{myGPIO\_mask} pendingInterrupt = \hyperlink{group__bare-metal_ga6115bde39f860d4e76e7d8f421ce222c}{myGPIO\_PendingPinInterrupt}(&
      \hyperlink{interrupt__bare_8c_ac2f1233c00752afeb2ecfd305c5fe36a}{swc\_gpio});
    \hyperlink{group__bare-metal_gab6ad3dda867515825890c97dbf6f55db}{myGPIO\_PinInterruptAck}(&\hyperlink{interrupt__bare_8c_ac2f1233c00752afeb2ecfd305c5fe36a}{swc\_gpio}, pendingInterrupt);

    \hyperlink{group__bare-metal_ga402a0d20afc0cb7c25554b8b023f4253}{myGPIO\_mask} value = \hyperlink{group__bare-metal_gac35776cd6652f7b932a132f3f6959a11}{myGPIO\_GetRead}(&\hyperlink{interrupt__bare_8c_ac2f1233c00752afeb2ecfd305c5fe36a}{swc\_gpio});
    \hyperlink{group__bare-metal_ga9d9ce9d2db7d77a588da4a3749f2f24d}{myGPIO\_SetValue}(&\hyperlink{interrupt__bare_8c_ac523aaaf082570c199faf2a98e03c219}{led\_gpio}, \hyperlink{group__bare-metal_gga402a0d20afc0cb7c25554b8b023f4253a6db6fa7be955ae379f543d96122e23a9}{myGPIO\_pin0} | 
      \hyperlink{group__bare-metal_gga402a0d20afc0cb7c25554b8b023f4253a1de6bdcc01efca2c39f584f5a20293be}{myGPIO\_pin1} | \hyperlink{group__bare-metal_gga402a0d20afc0cb7c25554b8b023f4253a1fb3f52d920ac8ba17b74dd73c27d783}{myGPIO\_pin2} | \hyperlink{group__bare-metal_gga402a0d20afc0cb7c25554b8b023f4253a4514d64390392b626aa4dbfaac8dc1e5}{myGPIO\_pin3}, 
      \hyperlink{group__bare-metal_ggaf634fe4a0e1eab8da5000b72d6ad362ba98cde80dbda025bd1ae7231c76b55674}{myGPIO\_reset});
    \hyperlink{group__bare-metal_ga9d9ce9d2db7d77a588da4a3749f2f24d}{myGPIO\_SetValue}(&\hyperlink{interrupt__bare_8c_ac523aaaf082570c199faf2a98e03c219}{led\_gpio}, value, \hyperlink{group__bare-metal_ggaf634fe4a0e1eab8da5000b72d6ad362ba10d296f3711d01189cc6c2d87f7c9149}{myGPIO\_set});

    \hyperlink{group__bare-metal_ga116e3a1077a317e9e42ded6dd4df64af}{myGPIO\_PinInterruptEnable}(&\hyperlink{interrupt__bare_8c_ac2f1233c00752afeb2ecfd305c5fe36a}{swc\_gpio}, enabledInterrut);
    \hyperlink{group__bare-metal_gada93ef6a9818e634f0a233ce14582216}{myGPIO\_GlobalInterruptEnable}(&\hyperlink{interrupt__bare_8c_ac2f1233c00752afeb2ecfd305c5fe36a}{swc\_gpio});
\}
\end{DoxyCode}
 La funzione di cui sopra non fa altro che disabilitare momentaneamente le interruzioni della periferica, leggere lo stato del registro “read”, resettare i led, per poi accendere solo quello corrispondente allo switch arrivo e riabilitare l'interrupt della periferica.

\subparagraph*{I\+S\+R per la gestione di interrupt provenienti dal gpio connesso ai button}


\begin{DoxyCode}
\textcolor{keywordtype}{void} \hyperlink{interrupt__bare_8c_aa4eac585cb67311e3e5fd374d6b09ad4}{btn\_isr}(\textcolor{keywordtype}{void}* data) \{
    \hyperlink{group__bare-metal_gaacca2871ac57a166e62bf431a2da7548}{myGPIO\_GlobalInterruptDisable}(&\hyperlink{interrupt__bare_8c_ac77d5df697b8a0d64704dee5f1433832}{btn\_gpio});
    \hyperlink{group__bare-metal_ga402a0d20afc0cb7c25554b8b023f4253}{myGPIO\_mask} enabledInterrut = \hyperlink{group__bare-metal_ga80ef1cf3e9bd8bfd4d849a0f3b8e7b2c}{myGPIO\_EnabledPinInterrupt}(&
      \hyperlink{interrupt__bare_8c_ac77d5df697b8a0d64704dee5f1433832}{btn\_gpio});
    \hyperlink{group__bare-metal_ga37d3df33ac50387d6f2e1fb5e2b13e49}{myGPIO\_PinInterruptDisable}(&\hyperlink{interrupt__bare_8c_ac77d5df697b8a0d64704dee5f1433832}{btn\_gpio}, enabledInterrut);

    \hyperlink{group__bare-metal_ga402a0d20afc0cb7c25554b8b023f4253}{myGPIO\_mask} pendingInterrupt = \hyperlink{group__bare-metal_ga6115bde39f860d4e76e7d8f421ce222c}{myGPIO\_PendingPinInterrupt}(&
      \hyperlink{interrupt__bare_8c_ac77d5df697b8a0d64704dee5f1433832}{btn\_gpio});
    \hyperlink{group__bare-metal_gab6ad3dda867515825890c97dbf6f55db}{myGPIO\_PinInterruptAck}(&\hyperlink{interrupt__bare_8c_ac77d5df697b8a0d64704dee5f1433832}{btn\_gpio}, pendingInterrupt);

    \hyperlink{group__bare-metal_ga402a0d20afc0cb7c25554b8b023f4253}{myGPIO\_mask} value = \hyperlink{group__bare-metal_gac35776cd6652f7b932a132f3f6959a11}{myGPIO\_GetRead}(&\hyperlink{interrupt__bare_8c_ac77d5df697b8a0d64704dee5f1433832}{btn\_gpio});
    \hyperlink{group__bare-metal_ga9d9ce9d2db7d77a588da4a3749f2f24d}{myGPIO\_SetValue}(&\hyperlink{interrupt__bare_8c_ac523aaaf082570c199faf2a98e03c219}{led\_gpio}, \hyperlink{group__bare-metal_gga402a0d20afc0cb7c25554b8b023f4253a6db6fa7be955ae379f543d96122e23a9}{myGPIO\_pin0} | 
      \hyperlink{group__bare-metal_gga402a0d20afc0cb7c25554b8b023f4253a1de6bdcc01efca2c39f584f5a20293be}{myGPIO\_pin1} | \hyperlink{group__bare-metal_gga402a0d20afc0cb7c25554b8b023f4253a1fb3f52d920ac8ba17b74dd73c27d783}{myGPIO\_pin2} | \hyperlink{group__bare-metal_gga402a0d20afc0cb7c25554b8b023f4253a4514d64390392b626aa4dbfaac8dc1e5}{myGPIO\_pin3}, 
      \hyperlink{group__bare-metal_ggaf634fe4a0e1eab8da5000b72d6ad362ba98cde80dbda025bd1ae7231c76b55674}{myGPIO\_reset});
    \hyperlink{group__bare-metal_ga9d9ce9d2db7d77a588da4a3749f2f24d}{myGPIO\_SetValue}(&\hyperlink{interrupt__bare_8c_ac523aaaf082570c199faf2a98e03c219}{led\_gpio}, value, \hyperlink{group__bare-metal_ggaf634fe4a0e1eab8da5000b72d6ad362ba10d296f3711d01189cc6c2d87f7c9149}{myGPIO\_set});

    \hyperlink{group__bare-metal_ga116e3a1077a317e9e42ded6dd4df64af}{myGPIO\_PinInterruptEnable}(&\hyperlink{interrupt__bare_8c_ac77d5df697b8a0d64704dee5f1433832}{btn\_gpio}, enabledInterrut);
    \hyperlink{group__bare-metal_gada93ef6a9818e634f0a233ce14582216}{myGPIO\_GlobalInterruptEnable}(&\hyperlink{interrupt__bare_8c_ac77d5df697b8a0d64704dee5f1433832}{btn\_gpio});
\}
\end{DoxyCode}
 La funzione di cui sopra non fa altro che disabilitare momentaneamente le interruzioni della periferica, leggere lo stato del registro “read”, resettare i led, per poi accendere solo quello corrispondente al button premuto e riabilitare l'interrupt della periferica.

\subparagraph*{Configurazione del G\+I\+C e registrazione degli interrupt handler}


\begin{DoxyCode}
\textcolor{keywordtype}{int} \hyperlink{interrupt__bare_8c_ad4c208e7b28dadf641bc3ec5b290d87d}{int\_config}(\textcolor{keywordtype}{void}) \{
    \textcolor{comment}{// inizializza il driver del GIC}
    Xil\_ExceptionInit();

    \textcolor{comment}{// ottiene i parametri di configurazione del GIC, lo configura ed inizializza}
    \textcolor{comment}{// sintassi : XScuGic\_LookupConfig(GIC\_id)}
    \textcolor{comment}{// sintassi : XScuGic\_CfgInitialize(GIC\_ptr, config, cpu\_address)}
    XScuGic\_Config *IntcConfig = XScuGic\_LookupConfig(\hyperlink{interrupt__bare_8c_a22782ed5aaa2e8d89334d159d14753b5}{gic\_id});
    \textcolor{keywordflow}{if} (IntcConfig == NULL)
        \textcolor{keywordflow}{return} -1;
    \textcolor{keywordflow}{if} (XScuGic\_CfgInitialize(&\hyperlink{interrupt__bare_8c_a0ad1175dbe99ac3f36c814258ec4f8c6}{GIC}, IntcConfig, IntcConfig->CpuBaseAddress) != XST\_SUCCESS)
        \textcolor{keywordflow}{return} -1;

    \textcolor{comment}{// registra l'interrupt handler del GIC alla logica di gestione del processing-system}
    \textcolor{comment}{// sintassi : Xil\_ExceptionRegisterHandler(XIL\_EXCEPTION\_ID\_INT, handler, gic\_ptr)}
    Xil\_ExceptionRegisterHandler(XIL\_EXCEPTION\_ID\_INT,(Xil\_ExceptionHandler)XScuGic\_InterruptHandler, &
      \hyperlink{interrupt__bare_8c_a0ad1175dbe99ac3f36c814258ec4f8c6}{GIC});

    \textcolor{comment}{// registrazione degli handler}
    \textcolor{comment}{// le righe seguenti stabiliscono quale sia l'handler da chiamare e quali dati bisogna passargli}
    \textcolor{comment}{// qualora si manifesti una interruzione su una line di irq.}
    \textcolor{comment}{// sintassi : XScuGic\_Connect(GIC, irq\_line, handler, data)}
    \textcolor{keywordflow}{if} (XScuGic\_Connect(&\hyperlink{interrupt__bare_8c_a0ad1175dbe99ac3f36c814258ec4f8c6}{GIC}, \hyperlink{interrupt__bare_8c_ab5602e3672ec6d03f39d2229dbcb9f74}{btn\_irq\_line}, (Xil\_InterruptHandler)
      \hyperlink{interrupt__bare_8c_aa4eac585cb67311e3e5fd374d6b09ad4}{btn\_isr}, (\textcolor{keywordtype}{void}*)NULL) != XST\_SUCCESS)
        \textcolor{keywordflow}{return} -1;
    \textcolor{keywordflow}{if} (XScuGic\_Connect(&\hyperlink{interrupt__bare_8c_a0ad1175dbe99ac3f36c814258ec4f8c6}{GIC}, \hyperlink{interrupt__bare_8c_a48d05dc71b54a160af13c8e31e9000b1}{swc\_irq\_line}, (Xil\_InterruptHandler)
      \hyperlink{interrupt__bare_8c_ad05dc46b6c6da383d687c5116864b4ed}{swc\_isr}, (\textcolor{keywordtype}{void}*)NULL) != XST\_SUCCESS)
            \textcolor{keywordflow}{return} -1;

    \textcolor{comment}{// abilitazione degli interrupt sulle linee connesse alle periferiche}
    \textcolor{comment}{// sintassi: XScuGic\_Enable(GIC,irq\_line);}
    XScuGic\_Enable(&\hyperlink{interrupt__bare_8c_a0ad1175dbe99ac3f36c814258ec4f8c6}{GIC}, \hyperlink{interrupt__bare_8c_ab5602e3672ec6d03f39d2229dbcb9f74}{btn\_irq\_line});
    XScuGic\_Enable(&\hyperlink{interrupt__bare_8c_a0ad1175dbe99ac3f36c814258ec4f8c6}{GIC}, \hyperlink{interrupt__bare_8c_a48d05dc71b54a160af13c8e31e9000b1}{swc\_irq\_line});

    \textcolor{comment}{// abilitazione degli interrupt delle periferiche}
    \hyperlink{group__bare-metal_gada93ef6a9818e634f0a233ce14582216}{myGPIO\_GlobalInterruptEnable}(&\hyperlink{interrupt__bare_8c_ac77d5df697b8a0d64704dee5f1433832}{btn\_gpio});
    \hyperlink{group__bare-metal_ga116e3a1077a317e9e42ded6dd4df64af}{myGPIO\_PinInterruptEnable}(&\hyperlink{interrupt__bare_8c_ac77d5df697b8a0d64704dee5f1433832}{btn\_gpio}, 
      \hyperlink{group__bare-metal_gga402a0d20afc0cb7c25554b8b023f4253a6db6fa7be955ae379f543d96122e23a9}{myGPIO\_pin0} | \hyperlink{group__bare-metal_gga402a0d20afc0cb7c25554b8b023f4253a1de6bdcc01efca2c39f584f5a20293be}{myGPIO\_pin1} | \hyperlink{group__bare-metal_gga402a0d20afc0cb7c25554b8b023f4253a1fb3f52d920ac8ba17b74dd73c27d783}{myGPIO\_pin2} | 
      \hyperlink{group__bare-metal_gga402a0d20afc0cb7c25554b8b023f4253a4514d64390392b626aa4dbfaac8dc1e5}{myGPIO\_pin3});
    \hyperlink{group__bare-metal_gada93ef6a9818e634f0a233ce14582216}{myGPIO\_GlobalInterruptEnable}(&\hyperlink{interrupt__bare_8c_ac2f1233c00752afeb2ecfd305c5fe36a}{swc\_gpio});
    \hyperlink{group__bare-metal_ga116e3a1077a317e9e42ded6dd4df64af}{myGPIO\_PinInterruptEnable}(&\hyperlink{interrupt__bare_8c_ac2f1233c00752afeb2ecfd305c5fe36a}{swc\_gpio}, 
      \hyperlink{group__bare-metal_gga402a0d20afc0cb7c25554b8b023f4253a6db6fa7be955ae379f543d96122e23a9}{myGPIO\_pin0} | \hyperlink{group__bare-metal_gga402a0d20afc0cb7c25554b8b023f4253a1de6bdcc01efca2c39f584f5a20293be}{myGPIO\_pin1} | \hyperlink{group__bare-metal_gga402a0d20afc0cb7c25554b8b023f4253a1fb3f52d920ac8ba17b74dd73c27d783}{myGPIO\_pin2} | 
      \hyperlink{group__bare-metal_gga402a0d20afc0cb7c25554b8b023f4253a4514d64390392b626aa4dbfaac8dc1e5}{myGPIO\_pin3});

    \textcolor{comment}{// abilitazione degli interrupt del processing-system}
    Xil\_ExceptionEnable();
    \textcolor{keywordflow}{return} 0;
\}
\end{DoxyCode}
 La funzione di configurazione delle interrupt e del G\+I\+C fa uso di alcune funzioni di libreria definite nell'header file \char`\"{}xscugic.\+h\char`\"{}, il quale implementa il driver della periferica G\+I\+C, e di alcune macro definite nel file \char`\"{}xparameters.\+h\char`\"{}. Solo per comodità le macro definite in xparameters.\+h sono state ridefinite come segue. 
\begin{DoxyCode}
\textcolor{preprocessor}{#define led\_base\_addr XPAR\_MYGPIO\_0\_S00\_AXI\_BASEADDR}
\textcolor{preprocessor}{#define btn\_base\_addr XPAR\_MYGPIO\_1\_S00\_AXI\_BASEADDR}
\textcolor{preprocessor}{#define swc\_base\_addr XPAR\_MYGPIO\_2\_S00\_AXI\_BASEADDR}
\textcolor{preprocessor}{#define led\_irq\_line XPAR\_FABRIC\_MYGPIO\_0\_INTERRUPT\_INTR}
\textcolor{preprocessor}{#define btn\_irq\_line XPAR\_FABRIC\_MYGPIO\_1\_INTERRUPT\_INTR}
\textcolor{preprocessor}{#define swc\_irq\_line XPAR\_FABRIC\_MYGPIO\_2\_INTERRUPT\_INTR}
\textcolor{preprocessor}{#define gic\_id      XPAR\_SCUGIC\_0\_DEVICE\_ID}
\end{DoxyCode}
 Le funzioni di libreria usate sono riportate di seguito, assieme ad una breve descrizione tratta dalla documentazione interna.
\begin{DoxyItemize}
\item Xil\+\_\+\+Exception\+Init()\+: inizializza gli exception-\/handlers di tutti i processori. Per A\+R\+M Cortex A53, R5 ed A9 gli exception-\/handlers sono sono inizializzati staticamente, per cui questa funzione non fa niente. Viene mantenutaper prevenire errori in fase di compilazione e per garantire backward-\/compatibility.
\item X\+Scu\+Gic\+\_\+\+Lookup\+Config()\+: looksup della configurazione di un device, basandosi sull'identificativo univoco dello stesso, dalla tabella contenente le configurazioni di tutti i device. Prende in ingresso un parametro Device\+Id e restituisce un puntatore a X\+Scu\+Gic, contenente la configurazione, o N\+U\+L\+L se il device non viene trovato.
\item X\+Scu\+Gic\+\_\+\+Cfg\+Initialize()\+: inizializza e configura un interrupt-\/controller instance/driver. La procedura di inizializzazione prevede\+:
\begin{DoxyItemize}
\item l'inizializzazione dei campi di una struttura X\+Scu\+Gic;
\item la configurazione della Initial vector-\/table, con funzioni stub;
\item disabilitazione di tutte le sorgenti di interruzione
\end{DoxyItemize}Parametri\+:
\begin{DoxyItemize}
\item Instance\+Ptr\+: puntatore a struttura X\+Scu\+Gic;
\item Config\+Ptr\+: puntatore alla configurazione del device, restituito dalla funzione X\+Scu\+Gic\+\_\+\+Lookup\+Config();
\item Effective\+Addr\+: indirizzo base del device; Restituisce X\+S\+T\+\_\+\+S\+U\+C\+C\+E\+S\+S se l'inizializzazione viene completata con successo.
\end{DoxyItemize}
\item Xil\+\_\+\+Exception\+Register\+Handler()\+: crea una connessione tra l'identificativo di una sorgente di eccezioni e l'handler associato, in modo che l'handler venga eseguito qualora si manifestasse una eccezione. Prende i seguenti parametri\+:
\begin{DoxyItemize}
\item exception\+\_\+id\+: I\+D della sorgente di eccezioni;
\item Handler\+: puntatore alla funzione di servizio;
\item Data\+: puntatore ai dati da passare all'handler;
\end{DoxyItemize}
\item X\+Scu\+Gic\+\_\+\+Connect()\+: crea una connessione tra l'identificativo di una sorgente di interruzioni e l'handler associato, in modo che l'handler venga eseguito qualora si manifestasse una interruzione. Prende i seguenti parametri\+:
\begin{DoxyItemize}
\item Instance\+Ptr\+: puntatore ad una istanza X\+Scu\+Gic;
\item Int\+\_\+\+Id\+: I\+D della sorgente di interruzioni;
\item Handler\+: puntatore alla funzione di servizio;
\item Call\+Back\+Ref\+: puntatore ai dati da passare alla isr;
\end{DoxyItemize}Restituisce X\+S\+T\+\_\+\+S\+U\+C\+C\+E\+S\+S se l'handler è stato connesso correttamente.
\item X\+Scu\+Gic\+\_\+\+Enable()\+: abilita la sorgente di interruzioni Int\+\_\+\+Id. Se ci sono pending interrupt per tale linea, scateneranno una interruzione dopo la chiamata a questa funzione. Parametri\+:
\begin{DoxyItemize}
\item Instance\+Ptr\+: puntatore ad una istanza X\+Scu\+Gic;
\item Int\+\_\+\+Id\+: I\+D della sorgente di interruzioni;
\end{DoxyItemize}
\item Xil\+\_\+\+Exception\+Enable()\+: abilita le interruzioni.
\end{DoxyItemize}


\begin{DoxyCodeInclude}
\end{DoxyCodeInclude}
 