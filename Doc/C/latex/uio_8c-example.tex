\hypertarget{uio_8c-example}{}\section{uio.\+c}
Il file \hyperlink{uio_8c}{uio.\+c} è un programma di esempio per l\textquotesingle{}interfacciamento con una periferica my\+G\+P\+IO. L\textquotesingle{}esempio mostra come possa, un programma userspace in esecuzione su sistema operativo Linux, interfacciarsi con un device my\+G\+P\+IO, interagendo con esso attraverso il driver U\+IO.

\begin{DoxyWarning}{Avvertimento}
Se nel device tree source non viene indicato \begin{center}compatible = \char`\"{}generic-\/uio\char`\"{};\end{center}  tra i driver compatibili con il device, il driver U\+IO non viene correttamente istanziato ed il programma non funzionerà.
\end{DoxyWarning}

\begin{DoxyCodeInclude}

\textcolor{preprocessor}{#include <inttypes.h>}
\textcolor{preprocessor}{#include <stdio.h>}
\textcolor{preprocessor}{#include <stdlib.h>}
\textcolor{preprocessor}{#include <unistd.h>}
\textcolor{preprocessor}{#include <sys/mman.h>}
\textcolor{preprocessor}{#include <fcntl.h>}
\textcolor{preprocessor}{#include "\hyperlink{my_g_p_i_o_8h}{myGPIO.h}"}

\textcolor{keywordtype}{void} \hyperlink{uio_8c_a05909651fa170a63e98e3f8e13451b7b}{howto}(\textcolor{keywordtype}{void}) \{
    printf(\textcolor{stringliteral}{"Uso:\(\backslash\)n"});
    printf(\textcolor{stringliteral}{"uio -d /dev/uioX -w|m <hex-value> -r\(\backslash\)n"});
    printf(\textcolor{stringliteral}{"\(\backslash\)t-m <hex-value>: scrive nel registro \(\backslash\)"mode\(\backslash\)"\(\backslash\)n"});
    printf(\textcolor{stringliteral}{"\(\backslash\)t-w <hex-value>: scrive nel registro \(\backslash\)"write\(\backslash\)"\(\backslash\)n"});
    printf(\textcolor{stringliteral}{"\(\backslash\)t-r: legge il valore del registro \(\backslash\)"read\(\backslash\)"\(\backslash\)n"});
    printf(\textcolor{stringliteral}{"I parametri possono anche essere usati assieme.\(\backslash\)n"});
\}

\textcolor{keywordtype}{int} \hyperlink{uio_8c_ab6b18eb1bf7bc996599c06dc6dad8f53}{parse\_args}(   \textcolor{keywordtype}{int}         argc,
                \textcolor{keywordtype}{char}        **argv,
                \textcolor{keywordtype}{char}        **uio,          \textcolor{comment}{// file uio da usare}
                uint8\_t     *op\_mode,       \textcolor{comment}{// impostato ad 1 se l'utente intende effettuare scrittuara su
       mode}
                uint32\_t    *mode\_value,    \textcolor{comment}{// valore che l'utente intende scrivere nel registro mode}
                uint8\_t     *op\_write,      \textcolor{comment}{// impostato ad 1 se l'utente intende effettuare scrittuara su
       write}
                uint32\_t    *write\_value,   \textcolor{comment}{// valore che l'utente intende scrivere nel registro write}
                uint8\_t     *op\_read)       \textcolor{comment}{// impostato ad 1 se l'utente intende effettuare lettura da
       read}
\{
    \textcolor{keywordtype}{int} par;
    \textcolor{keywordflow}{while}((par = getopt(argc, argv, \textcolor{stringliteral}{"d:w:m:r"})) != -1) \{
        \textcolor{keywordflow}{switch} (par) \{
        \textcolor{keywordflow}{case} \textcolor{charliteral}{'d'} :
            *uio = optarg;
            \textcolor{keywordflow}{break};
        \textcolor{keywordflow}{case} \textcolor{charliteral}{'w'} :
            *write\_value = strtoul(optarg, NULL, 0);
            *op\_write = 1;
            \textcolor{keywordflow}{break};
        \textcolor{keywordflow}{case} \textcolor{charliteral}{'m'} :
            *mode\_value = strtoul(optarg, NULL, 0);
            *op\_mode = 1;
            \textcolor{keywordflow}{break};
        \textcolor{keywordflow}{case} \textcolor{charliteral}{'r'} :
            *op\_read = 1;
            \textcolor{keywordflow}{break};
        default :
            printf(\textcolor{stringliteral}{"%c: parametro sconosciuto.\(\backslash\)n"}, par);
            \hyperlink{uio_8c_a05909651fa170a63e98e3f8e13451b7b}{howto}();
            \textcolor{keywordflow}{return} -1;
        \}
    \}
    \textcolor{keywordflow}{return} 0;
\}


\textcolor{keywordtype}{void} \hyperlink{uio_8c_a879d8b839631449ecb5bc4d0721432b6}{gpio\_op} (   \textcolor{keywordtype}{void}*       vrt\_gpio\_addr,
                uint8\_t     op\_mode,
                uint32\_t    mode\_value,
                uint8\_t     op\_write,
                uint32\_t    write\_value,
                uint8\_t     op\_read)
\{
    printf(\textcolor{stringliteral}{"Indirizzo gpio: %08x\(\backslash\)n"}, (uint32\_t)vrt\_gpio\_addr);

\textcolor{preprocessor}{    #ifdef \_\_XIL\_GPIO\_\_}
\textcolor{preprocessor}{#define MODE\_OFFSET     4U}
\textcolor{preprocessor}{#define WRITE\_OFFSET    0U}
\textcolor{preprocessor}{#define READ\_OFFSET     8U}
\textcolor{preprocessor}{#else}
    \hyperlink{structmy_g_p_i_o__t}{myGPIO\_t} gpio;
    \hyperlink{group__bare-metal_ga588201358d1633c53535b288c9198531}{myGPIO\_Init}(&gpio, (uint32\_t)vrt\_gpio\_addr);
\textcolor{preprocessor}{#endif}

    \textcolor{keywordflow}{if} (op\_mode == 1) \{
\textcolor{preprocessor}{#ifdef \_\_XIL\_GPIO\_\_}
        *((uint32\_t*)(vrt\_gpio\_addr+\hyperlink{mygpiok_8c_a7a63b24ca5489eb3206598e3d90fe19c}{MODE\_OFFSET})) = mode\_value;
        mode\_value = *((uint32\_t*)(vrt\_gpio\_addr+\hyperlink{mygpiok_8c_a7a63b24ca5489eb3206598e3d90fe19c}{MODE\_OFFSET}));
\textcolor{preprocessor}{#else}
        \hyperlink{group__bare-metal_ga43e82eb0febd452635a438fbd9cb853b}{myGPIO\_SetMode}(&gpio, mode\_value, \hyperlink{group__bare-metal_gga76b849f0e0c05e7f9161bdb33396f2b1a2d66976280eb7595a42c631683bdfad6}{myGPIO\_write});
        \hyperlink{group__bare-metal_ga43e82eb0febd452635a438fbd9cb853b}{myGPIO\_SetMode}(&gpio, ~mode\_value, \hyperlink{group__bare-metal_ggaf634fe4a0e1eab8da5000b72d6ad362ba98cde80dbda025bd1ae7231c76b55674}{myGPIO\_reset});
\textcolor{preprocessor}{#endif}
        printf(\textcolor{stringliteral}{"Scrittura sul registro mode: %08x\(\backslash\)n"}, mode\_value);
    \}
    \textcolor{keywordflow}{if} (op\_write == 1) \{
\textcolor{preprocessor}{#ifdef \_\_XIL\_GPIO\_\_}
        *((uint32\_t*)(vrt\_gpio\_addr+\hyperlink{mygpiok_8c_a77d96306ed0e813f93c4c3f98b970b86}{WRITE\_OFFSET})) = write\_value;
        write\_value = *((uint32\_t*)(vrt\_gpio\_addr+\hyperlink{mygpiok_8c_a77d96306ed0e813f93c4c3f98b970b86}{WRITE\_OFFSET}));
\textcolor{preprocessor}{#else}
        \hyperlink{group__bare-metal_ga9d9ce9d2db7d77a588da4a3749f2f24d}{myGPIO\_SetValue}(&gpio, write\_value, \hyperlink{group__bare-metal_ggaf634fe4a0e1eab8da5000b72d6ad362ba10d296f3711d01189cc6c2d87f7c9149}{myGPIO\_set});
        \hyperlink{group__bare-metal_ga9d9ce9d2db7d77a588da4a3749f2f24d}{myGPIO\_SetValue}(&gpio, ~write\_value, \hyperlink{group__bare-metal_ggaf634fe4a0e1eab8da5000b72d6ad362ba98cde80dbda025bd1ae7231c76b55674}{myGPIO\_reset});
\textcolor{preprocessor}{#endif}
        printf(\textcolor{stringliteral}{"Scrittura sul registro write: %08x\(\backslash\)n"}, write\_value);
    \}
    \textcolor{keywordflow}{if} (op\_read == 1) \{
        uint32\_t read\_value = 0;
\textcolor{preprocessor}{#ifdef \_\_XIL\_GPIO\_\_}
        read\_value = *((uint32\_t*)(vrt\_gpio\_addr+\hyperlink{mygpiok_8c_ad32c3a5b42163e171daccde5b5d5de02}{READ\_OFFSET}));
\textcolor{preprocessor}{#else}
        read\_value = \hyperlink{group__bare-metal_gac35776cd6652f7b932a132f3f6959a11}{myGPIO\_GetRead}(&gpio);
\textcolor{preprocessor}{#endif}
        printf(\textcolor{stringliteral}{"Lettura dat registro read: %08x\(\backslash\)n"}, read\_value);
    \}
\}

\textcolor{keywordtype}{int} \hyperlink{uio_8c_a3c04138a5bfe5d72780bb7e82a18e627}{main}(\textcolor{keywordtype}{int} argc, \textcolor{keywordtype}{char}** argv) \{
    \textcolor{keywordtype}{char}* uio\_file = 0;         \textcolor{comment}{// nome del file uio}
    uint8\_t op\_mode = 0;        \textcolor{comment}{// impostato ad 1 se l'utente intende effettuare scrittuara su mode}
    uint32\_t mode\_value;        \textcolor{comment}{// valore che l'utente intende scrivere nel registro mode}
    uint8\_t op\_write = 0;       \textcolor{comment}{// impostato ad 1 se l'utente intende effettuare scrittuara su write}
    uint32\_t write\_value;       \textcolor{comment}{// valore che l'utente intende scrivere nel registro write}
    uint8\_t op\_read = 0;        \textcolor{comment}{// impostato ad 1 se l'utente intende effettuare lettura da read}

    printf(\textcolor{stringliteral}{"%s build %d\(\backslash\)n"}, argv[0], BUILD); \textcolor{comment}{// BUILD viene definita in compilazione}

    \textcolor{keywordflow}{if} (\hyperlink{uio_8c_ab6b18eb1bf7bc996599c06dc6dad8f53}{parse\_args}(argc, argv, &uio\_file, &op\_mode, &mode\_value, &op\_write, &write\_value, &
      op\_read) == -1)
        \textcolor{keywordflow}{return} -1;
    \textcolor{keywordflow}{if} (uio\_file == 0) \{
        printf(\textcolor{stringliteral}{"è necessario specificare il device UIO col quale interagire.\(\backslash\)n"});
        \hyperlink{uio_8c_a05909651fa170a63e98e3f8e13451b7b}{howto}();
        \textcolor{keywordflow}{return} -1;
    \}
    \textcolor{keywordtype}{int} descriptor = open (uio\_file, O\_RDWR);
    \textcolor{keywordflow}{if} (descriptor < 1) \{
        perror(argv[0]);
        \textcolor{keywordflow}{return} -1;
    \}
    uint32\_t page\_size = sysconf(\_SC\_PAGESIZE);     \textcolor{comment}{// dimensione della pagina}
    \textcolor{keywordtype}{void}* vrt\_gpio\_addr = mmap(NULL, page\_size, PROT\_READ | PROT\_WRITE, MAP\_SHARED, descriptor, 0);
    \textcolor{keywordflow}{if} (vrt\_gpio\_addr == MAP\_FAILED) \{
        printf(\textcolor{stringliteral}{"Mapping indirizzo fisico - indirizzo virtuale FALLITO!\(\backslash\)n"});
        \textcolor{keywordflow}{return} -1;
    \}
    \hyperlink{uio_8c_a879d8b839631449ecb5bc4d0721432b6}{gpio\_op}(vrt\_gpio\_addr, op\_mode, mode\_value, op\_write, write\_value, op\_read);

    munmap(vrt\_gpio\_addr, page\_size);
    close(descriptor);

    \textcolor{keywordflow}{return} 0;
\}
\end{DoxyCodeInclude}
 