\hypertarget{uio-int_8c-example}{\section{uio-\/int.\+c}
}
Questo è un programma di esempio per l'interfacciamento con una periferica my\+G\+P\+I\+O.\begin{DoxyAuthor}{Autore}
Salvatore Barone \href{mailto:salvator.barone@gmail.com}{\tt salvator.\+barone@gmail.\+com} 
\end{DoxyAuthor}
\begin{DoxyDate}{Data}
14 06 2017
\end{DoxyDate}
\begin{DoxyCopyright}{Copyright}
Copyright 2017 Salvatore Barone \href{mailto:salvator.barone@gmail.com}{\tt salvator.\+barone@gmail.\+com}
\end{DoxyCopyright}
This file is part of Zynq7000\+Driver\+Pack

Zynq7000\+Driver\+Pack is free software; you can redistribute it and/or modify it under the terms of the G\+N\+U General Public License as published by the Free Software Foundation; either version 3 of the License, or any later version.

Zynq7000\+Driver\+Pack is distributed in the hope that it will be useful, but W\+I\+T\+H\+O\+U\+T A\+N\+Y W\+A\+R\+R\+A\+N\+T\+Y; without even the implied warranty of M\+E\+R\+C\+H\+A\+N\+T\+A\+B\+I\+L\+I\+T\+Y or F\+I\+T\+N\+E\+S\+S F\+O\+R A P\+A\+R\+T\+I\+C\+U\+L\+A\+R P\+U\+R\+P\+O\+S\+E. See the G\+N\+U General Public License for more details.

You should have received a copy of the G\+N\+U General Public License along with this program; if not, write to the Free Software Foundation, Inc., 51 Franklin Street, Fifth Floor, Boston, M\+A 02110-\/1301, U\+S\+A.

In questo specifico esempio l'interfacciamento avviene da user-\/space, interagendo attraverso il driver uio. {\bfseries Utilizza gli interrupt per la lettura}.

\begin{DoxyWarning}{Avvertimento}
Se nel device tree source non viene indicato \begin{center}compatible = \char`\"{}generic-\/uio\char`\"{};\end{center}  tra i driver compatibili con il device, il driver U\+I\+O non viene correttamente istanziato ed il programma non funzionerà.
\end{DoxyWarning}

\begin{DoxyCodeInclude}

\textcolor{preprocessor}{#include <inttypes.h>}
\textcolor{preprocessor}{#include <stdio.h>}
\textcolor{preprocessor}{#include <stdlib.h>}
\textcolor{preprocessor}{#include <unistd.h>}
\textcolor{preprocessor}{#include <sys/mman.h>}
\textcolor{preprocessor}{#include <fcntl.h>}
\textcolor{preprocessor}{#include "\hyperlink{my_g_p_i_o_8h}{myGPIO.h}"}
\textcolor{preprocessor}{#include "\hyperlink{xil__gpio_8h}{xil\_gpio.h}"}

\textcolor{keywordtype}{void} \hyperlink{uio-int_8c_a05909651fa170a63e98e3f8e13451b7b}{howto}(\textcolor{keywordtype}{void}) \{
    printf(\textcolor{stringliteral}{"Uso:\(\backslash\)n"});
    printf(\textcolor{stringliteral}{"uio -d /dev/uioX -w|m <hex-value> -r\(\backslash\)n"});
    printf(\textcolor{stringliteral}{"\(\backslash\)t-m <hex-value>: scrive nel registro \(\backslash\)"mode\(\backslash\)"\(\backslash\)n"});
    printf(\textcolor{stringliteral}{"\(\backslash\)t-w <hex-value>: scrive nel registro \(\backslash\)"write\(\backslash\)"\(\backslash\)n"});
    printf(\textcolor{stringliteral}{"\(\backslash\)t-r: legge il valore del registro \(\backslash\)"read\(\backslash\)"\(\backslash\)n"});
    printf(\textcolor{stringliteral}{"I parametri possono anche essere usati assieme.\(\backslash\)n"});
\}


\textcolor{keywordtype}{int} \hyperlink{uio-int_8c_ab6b18eb1bf7bc996599c06dc6dad8f53}{parse\_args}(   \textcolor{keywordtype}{int}         argc,
                \textcolor{keywordtype}{char}        **argv,
                \textcolor{keywordtype}{char}        **uio,
                uint8\_t     *op\_mode,
                uint32\_t    *mode\_value,
                uint8\_t     *op\_write,
                uint32\_t    *write\_value,
                uint8\_t     *op\_read)
\{
    \textcolor{keywordtype}{int} par;
    \textcolor{keywordflow}{while}((par = getopt(argc, argv, \textcolor{stringliteral}{"d:w:m:r"})) != -1) \{
        \textcolor{keywordflow}{switch} (par) \{
        \textcolor{keywordflow}{case} \textcolor{charliteral}{'d'} :
            *uio = optarg;
            \textcolor{keywordflow}{break};
        \textcolor{keywordflow}{case} \textcolor{charliteral}{'w'} :
            *write\_value = strtoul(optarg, NULL, 0);
            *op\_write = 1;
            \textcolor{keywordflow}{break};
        \textcolor{keywordflow}{case} \textcolor{charliteral}{'m'} :
            *mode\_value = strtoul(optarg, NULL, 0);
            *op\_mode = 1;
            \textcolor{keywordflow}{break};
        \textcolor{keywordflow}{case} \textcolor{charliteral}{'r'} :
            *op\_read = 1;
            \textcolor{keywordflow}{break};
        \textcolor{keywordflow}{default} :
            printf(\textcolor{stringliteral}{"%c: parametro sconosciuto.\(\backslash\)n"}, par);
            \hyperlink{uio-int_8c_a05909651fa170a63e98e3f8e13451b7b}{howto}();
            \textcolor{keywordflow}{return} -1;
        \}
    \}
    \textcolor{keywordflow}{return} 0;
\}


\textcolor{keywordtype}{void} \hyperlink{uio-int_8c_a78b676750c5d08c316cad35ec3963c53}{gpio\_op} (   \textcolor{keywordtype}{void}*       vrt\_gpio\_addr,
                \textcolor{keywordtype}{int}         uio\_descriptor,
                uint8\_t     op\_mode,
                uint32\_t    mode\_value,
                uint8\_t     op\_write,
                uint32\_t    write\_value,
                uint8\_t     op\_read)
\{
    printf(\textcolor{stringliteral}{"Indirizzo gpio: %08x\(\backslash\)n"}, (uint32\_t)vrt\_gpio\_addr);
\textcolor{preprocessor}{#ifndef \_\_XIL\_GPIO\_\_}
    \hyperlink{structmy_g_p_i_o__t}{myGPIO\_t} gpio;
    \hyperlink{group__bare-metal_ga588201358d1633c53535b288c9198531}{myGPIO\_Init}(&gpio, vrt\_gpio\_addr);
\textcolor{preprocessor}{#endif}

    \textcolor{keywordflow}{if} (op\_mode == 1) \{
\textcolor{preprocessor}{#ifdef \_\_XIL\_GPIO\_\_}
        *((uint32\_t*)(vrt\_gpio\_addr+\hyperlink{xil__gpio_8h_a6e2a77c24e8a4e2b3800085ac10a8cf6}{GPIO\_TRI\_OFFSET})) = mode\_value;
        mode\_value = *((uint32\_t*)(vrt\_gpio\_addr+\hyperlink{xil__gpio_8h_a6e2a77c24e8a4e2b3800085ac10a8cf6}{GPIO\_TRI\_OFFSET}));
\textcolor{preprocessor}{#else}
        \hyperlink{group__bare-metal_ga43e82eb0febd452635a438fbd9cb853b}{myGPIO\_SetMode}(&gpio, mode\_value, \hyperlink{group__bare-metal_gga76b849f0e0c05e7f9161bdb33396f2b1a2d66976280eb7595a42c631683bdfad6}{myGPIO\_write});
        \hyperlink{group__bare-metal_ga43e82eb0febd452635a438fbd9cb853b}{myGPIO\_SetMode}(&gpio, ~mode\_value, \hyperlink{group__bare-metal_ggaf634fe4a0e1eab8da5000b72d6ad362ba98cde80dbda025bd1ae7231c76b55674}{myGPIO\_reset});
\textcolor{preprocessor}{#endif}
        printf(\textcolor{stringliteral}{"Scrittura sul registro mode: %08x\(\backslash\)n"}, mode\_value);
    \}
    \textcolor{keywordflow}{if} (op\_write == 1) \{
\textcolor{preprocessor}{#ifdef \_\_XIL\_GPIO\_\_}
        *((uint32\_t*)(vrt\_gpio\_addr+\hyperlink{xil__gpio_8h_a78907649b00f7076af686bcac4cd1b8c}{GPIO\_DATA\_OFFSET})) = write\_value;
        write\_value = *((uint32\_t*)(vrt\_gpio\_addr+\hyperlink{xil__gpio_8h_a78907649b00f7076af686bcac4cd1b8c}{GPIO\_DATA\_OFFSET}));
\textcolor{preprocessor}{#else}
        \hyperlink{group__bare-metal_ga9d9ce9d2db7d77a588da4a3749f2f24d}{myGPIO\_SetValue}(&gpio, write\_value, \hyperlink{group__bare-metal_ggaf634fe4a0e1eab8da5000b72d6ad362ba10d296f3711d01189cc6c2d87f7c9149}{myGPIO\_set});
        \hyperlink{group__bare-metal_ga9d9ce9d2db7d77a588da4a3749f2f24d}{myGPIO\_SetValue}(&gpio, ~write\_value, \hyperlink{group__bare-metal_ggaf634fe4a0e1eab8da5000b72d6ad362ba98cde80dbda025bd1ae7231c76b55674}{myGPIO\_reset});
\textcolor{preprocessor}{#endif}
        printf(\textcolor{stringliteral}{"Scrittura sul registro write: %08x\(\backslash\)n"}, write\_value);
    \}
    \textcolor{keywordflow}{if} (op\_read == 1) \{
        uint32\_t read\_value = 0;
        \textcolor{comment}{// interrupt enable (interni alla periferica)}
\textcolor{preprocessor}{        #ifdef \_\_XIL\_GPIO\_\_}
        \textcolor{comment}{// (globale + canale 2)}
        \hyperlink{xil__gpio_8c_aac9ff33f07964a1f5f9b8b1173072d67}{XilGpio\_Global\_Interrupt}((uint32\_t*)vrt\_gpio\_addr, 
      \hyperlink{xil__gpio_8h_adede88fc60bf8fc2dd1df41223bcaab2}{GLOBAL\_INTR\_ENABLE});
        \hyperlink{xil__gpio_8c_ab335ddab38389969b7a1fdb226eb1fbf}{XilGpio\_Channel\_Interrupt}((uint32\_t*)vrt\_gpio\_addr, 
      \hyperlink{xil__gpio_8h_a09e472673cd2e996bc6cb7d60fea9b9b}{CHANNEL2\_INTR\_ENABLE});
\textcolor{preprocessor}{        #else}
        \hyperlink{group__bare-metal_gada93ef6a9818e634f0a233ce14582216}{myGPIO\_GlobalInterruptEnable}(&gpio);
        \hyperlink{group__bare-metal_ga116e3a1077a317e9e42ded6dd4df64af}{myGPIO\_PinInterruptEnable}(&gpio, \hyperlink{group__bare-metal_gga402a0d20afc0cb7c25554b8b023f4253a6db6fa7be955ae379f543d96122e23a9}{myGPIO\_pin0}|
      \hyperlink{group__bare-metal_gga402a0d20afc0cb7c25554b8b023f4253a1de6bdcc01efca2c39f584f5a20293be}{myGPIO\_pin1}|\hyperlink{group__bare-metal_gga402a0d20afc0cb7c25554b8b023f4253a1fb3f52d920ac8ba17b74dd73c27d783}{myGPIO\_pin2}|\hyperlink{group__bare-metal_gga402a0d20afc0cb7c25554b8b023f4253a4514d64390392b626aa4dbfaac8dc1e5}{myGPIO\_pin3});
\textcolor{preprocessor}{        #endif}

        printf(\textcolor{stringliteral}{"Attesa dell'interruzione\(\backslash\)n"});
        uint32\_t interrupt\_count = 1;
        \textcolor{keywordflow}{if} (read(uio\_descriptor, &interrupt\_count, \textcolor{keyword}{sizeof}(uint32\_t)) != \textcolor{keyword}{sizeof}(uint32\_t)) \{
            printf(\textcolor{stringliteral}{"Read error!\(\backslash\)n"});
            \textcolor{keywordflow}{return};
        \}
        printf(\textcolor{stringliteral}{"Interrupt count: %08x\(\backslash\)n"}, interrupt\_count);
        \textcolor{comment}{// disabilitazione interrupt (interni alla periferica)}
\textcolor{preprocessor}{        #ifdef \_\_XIL\_GPIO\_\_}
        \hyperlink{xil__gpio_8c_aac9ff33f07964a1f5f9b8b1173072d67}{XilGpio\_Global\_Interrupt}((uint32\_t*)vrt\_gpio\_addr, 
      \hyperlink{xil__gpio_8h_a893a2062ab316b15fd6bdca35310b86d}{GLOBAL\_INTR\_DISABLE});
        \hyperlink{xil__gpio_8c_ab335ddab38389969b7a1fdb226eb1fbf}{XilGpio\_Channel\_Interrupt}((uint32\_t*)vrt\_gpio\_addr, 
      \hyperlink{xil__gpio_8h_a99ad674322d803e37f8233f207e3d1d1}{CHANNEL2\_INTR\_DISABLE});
\textcolor{preprocessor}{        #else}
        \hyperlink{group__bare-metal_gaacca2871ac57a166e62bf431a2da7548}{myGPIO\_GlobalInterruptDisable}(&gpio);
        \hyperlink{group__bare-metal_ga37d3df33ac50387d6f2e1fb5e2b13e49}{myGPIO\_PinInterruptDisable}(&gpio, \hyperlink{group__bare-metal_gga402a0d20afc0cb7c25554b8b023f4253a6db6fa7be955ae379f543d96122e23a9}{myGPIO\_pin0}|
      \hyperlink{group__bare-metal_gga402a0d20afc0cb7c25554b8b023f4253a1de6bdcc01efca2c39f584f5a20293be}{myGPIO\_pin1}|\hyperlink{group__bare-metal_gga402a0d20afc0cb7c25554b8b023f4253a1fb3f52d920ac8ba17b74dd73c27d783}{myGPIO\_pin2}|\hyperlink{group__bare-metal_gga402a0d20afc0cb7c25554b8b023f4253a4514d64390392b626aa4dbfaac8dc1e5}{myGPIO\_pin3});
\textcolor{preprocessor}{        #endif}

        \textcolor{comment}{// "servizio" dell'interruzione.}
        \textcolor{comment}{// lettura del registro}
\textcolor{preprocessor}{        #ifdef \_\_XIL\_GPIO\_\_}
        read\_value = *((uint32\_t*)(vrt\_gpio\_addr+\hyperlink{xil__gpio_8h_ae5cbf47902785bbcc678bddb8212c86b}{GPIO\_READ\_OFFSET}));
\textcolor{preprocessor}{        #else}
        read\_value = \hyperlink{group__bare-metal_gac35776cd6652f7b932a132f3f6959a11}{myGPIO\_GetRead}(&gpio);
\textcolor{preprocessor}{        #endif}
        printf(\textcolor{stringliteral}{"Lettura dat registro read: %08x\(\backslash\)n"}, read\_value);
\textcolor{preprocessor}{        #ifdef \_\_XIL\_GPIO\_\_}
        \textcolor{keywordflow}{while}(*((uint32\_t*)(vrt\_gpio\_addr+\hyperlink{xil__gpio_8h_ae5cbf47902785bbcc678bddb8212c86b}{GPIO\_READ\_OFFSET}))!=0);
\textcolor{preprocessor}{        #else}
        \textcolor{keywordflow}{while}(\hyperlink{group__bare-metal_gac35776cd6652f7b932a132f3f6959a11}{myGPIO\_GetRead}(&gpio) != 0);
\textcolor{preprocessor}{        #endif}

        \textcolor{comment}{// invio dell'ack alla periferica}
\textcolor{preprocessor}{        #ifdef \_\_XIL\_GPIO\_\_}
        \hyperlink{xil__gpio_8c_ad7b9691b1dc24679b60c25a6d5dc9647}{XilGpio\_Ack\_Interrupt}((uint32\_t*)vrt\_gpio\_addr, 
      \hyperlink{xil__gpio_8h_af56fe75e9fc0fcfc0b8fdda217d9d842}{CHANNEL2\_ACK});
\textcolor{preprocessor}{        #else}
        \hyperlink{group__bare-metal_gab6ad3dda867515825890c97dbf6f55db}{myGPIO\_PinInterruptAck}(&gpio, 
      \hyperlink{group__bare-metal_ga6115bde39f860d4e76e7d8f421ce222c}{myGPIO\_PendingPinInterrupt}(&gpio));
\textcolor{preprocessor}{        #endif}

        uint32\_t reenable = 1;
        \textcolor{keywordflow}{if} (write(uio\_descriptor, (\textcolor{keywordtype}{void}*)&reenable, \textcolor{keyword}{sizeof}(uint32\_t)) != \textcolor{keyword}{sizeof}(uint32\_t)) \{
            printf(\textcolor{stringliteral}{"Write error!\(\backslash\)n"});
            \textcolor{keywordflow}{return};
        \}
    \}
\}

\textcolor{keywordtype}{int} \hyperlink{uio-int_8c_a3c04138a5bfe5d72780bb7e82a18e627}{main}(\textcolor{keywordtype}{int} argc, \textcolor{keywordtype}{char}** argv) \{
    \textcolor{keywordtype}{char}* uio\_file = 0;         \textcolor{comment}{// nome del file uio}
    uint8\_t op\_mode = 0;        \textcolor{comment}{// impostato ad 1 se l'utente intende effettuare scrittuara su mode}
    uint32\_t mode\_value;        \textcolor{comment}{// valore che l'utente intende scrivere nel registro mode}
    uint8\_t op\_write = 0;       \textcolor{comment}{// impostato ad 1 se l'utente intende effettuare scrittuara su write}
    uint32\_t write\_value;       \textcolor{comment}{// valore che l'utente intende scrivere nel registro write}
    uint8\_t op\_read = 0;        \textcolor{comment}{// impostato ad 1 se l'utente intende effettuare lettura da read}

    printf(\textcolor{stringliteral}{"%s build %d\(\backslash\)n"}, argv[0], BUILD); \textcolor{comment}{// BUILD viene definita in compilazione}
\textcolor{comment}{}    \textcolor{keywordflow}{if} (\hyperlink{uio-int_8c_ab6b18eb1bf7bc996599c06dc6dad8f53}{parse\_args}(argc, argv, &uio\_file, &op\_mode, &mode\_value, &op\_write, &write\_value, &
      op\_read) == -1)
        \textcolor{keywordflow}{return} -1;
    \textcolor{keywordflow}{if} (uio\_file == 0) \{
        printf(\textcolor{stringliteral}{"è necessario specificare l'indirizzo di memoria del device.\(\backslash\)n"});
        \hyperlink{uio-int_8c_a05909651fa170a63e98e3f8e13451b7b}{howto}();
        \textcolor{keywordflow}{return} -1;
    \}
    \textcolor{keywordtype}{int} descriptor = open (uio\_file, O\_RDWR);
    \textcolor{keywordflow}{if} (descriptor < 1) \{
        perror(argv[0]);
        \textcolor{keywordflow}{return} -1;
    \}
    uint32\_t page\_size = sysconf(\_SC\_PAGESIZE);     \textcolor{comment}{// dimensione della pagina}
    \textcolor{keywordtype}{void}* vrt\_gpio\_addr = mmap(NULL, page\_size, PROT\_READ | PROT\_WRITE, MAP\_SHARED, descriptor, 0);
    \textcolor{keywordflow}{if} (vrt\_gpio\_addr == MAP\_FAILED) \{
        printf(\textcolor{stringliteral}{"Mapping indirizzo fisico - indirizzo virtuale FALLITO!\(\backslash\)n"});
        \textcolor{keywordflow}{return} -1;
    \}
    \hyperlink{uio-int_8c_a78b676750c5d08c316cad35ec3963c53}{gpio\_op}(vrt\_gpio\_addr, descriptor, op\_mode, mode\_value, op\_write, write\_value, op\_read);

    munmap(vrt\_gpio\_addr, page\_size);
    close(descriptor);

    \textcolor{keywordflow}{return} 0;
\}


\end{DoxyCodeInclude}
 